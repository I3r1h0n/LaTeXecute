\documentclass[11pt]{article}
\usepackage{amsmath}
\usepackage{amssymb}
\usepackage{amsfonts}
\usepackage{shellesc}
\usepackage{geometry}

\geometry{a4paper, margin=1in}

\title{Mathematical Discourse: A Treatise on Analytical Methods}
\author{Dr. Mathematics Scholar}
\date{\today}

\newcommand{\loremonepar}{Lorem ipsum dolor sit amet, consectetur adipiscing elit. Sed do eiusmod tempor incididunt ut labore et dolore magna aliqua. Ut enim ad minim veniam, quis nostrud exercitation ullamco laboris nisi ut aliquip ex ea commodo consequat.}
\newcommand{\loremtwopar}{Duis aute irure dolor in reprehenderit in voluptate velit esse cillum dolore eu fugiat nulla pariatur. Excepteur sint occaecat cupidatat non proident, sunt in culpa qui officia deserunt mollit anim id est laborum. Sed ut perspiciatis unde omnis iste natus error sit voluptatem accusantium doloremque laudantium.}
\newcommand{\loremthreepar}{At vero eos et accusamus et iusto odio dignissimos ducimus qui blanditiis praesentium voluptatum deleniti atque corrupti quos dolores et quas molestias excepturi sint occaecati cupiditate non provident, similique sunt in culpa qui officia deserunt mollitia animi.}

\begin{document}

\maketitle

\section{Introduction to Mathematical Analysis}

\loremonepar

Consider the fundamental equation of analysis:
\begin{equation}
    \mathcal{L}(f) = \int_{-\infty}^{\infty} e^{-s t} f(t)  dt
\end{equation}
where $f(t)$ represents an arbitrary function in the space of continuous distributions.

\loremtwopar

The generalized Fourier transform can be expressed as:
\begin{equation}
    F(\omega) = \frac{1}{\sqrt{2\pi}} \int_{-\infty}^{\infty} f(t) e^{-i\omega t}  dt
\end{equation}

\section{Advanced Theoretical Constructs}

\subsection{Matrix Theory Applications}

\loremthreepar

Given a symmetric matrix $A \in \mathbb{R}^{n \times n}$, we have:
\begin{equation}
    A = Q \Lambda Q^T = \sum_{i=1}^{n} \lambda_i q_i q_i^T
\end{equation}
where $\Lambda = \text{diag}(\lambda_1, \lambda_2, \dots, \lambda_n)$ contains the eigenvalues.

\loremonepar

The characteristic polynomial is defined as:
\begin{equation}
    p(\lambda) = \det(A - \lambda I) = \prod_{i=1}^{n} (\lambda_i - \lambda)
\end{equation}

\subsection{Probability and Stochastic Processes}

\loremtwopar

For a random variable $X$ with probability density function $f_X(x)$, the expected value is:
\begin{equation}
    \mathbb{E}[X] = \int_{-\infty}^{\infty} x f_X(x)  dx
\end{equation}

The cumulative distribution function follows:
\begin{equation}
    F_X(x) = P(X \leq x) = \int_{-\infty}^{x} f_X(t)  dt
\end{equation}

\loremthreepar

\section{Differential Equations and Their Solutions}

Consider the second-order linear differential equation:
\begin{equation}
    a(x)\frac{d^2 y}{dx^2} + b(x)\frac{dy}{dx} + c(x)y = f(x)
\end{equation}

\loremonepar

The homogeneous solution takes the form:
\begin{equation}
    y_h(x) = C_1 e^{r_1 x} + C_2 e^{r_2 x}
\end{equation}
where $r_1$ and $r_2$ are roots of the characteristic equation.

\subsection{Partial Differential Equations}

\loremtwopar

The heat equation in one dimension:
\begin{equation}
    \frac{\partial u}{\partial t} = \alpha \frac{\partial^2 u}{\partial x^2}
\end{equation}

The wave equation:
\begin{equation}
    \frac{\partial^2 u}{\partial t^2} = c^2 \frac{\partial^2 u}{\partial x^2}
\end{equation}

\loremthreepar

\section{Number Theory and Abstract Algebra}

\subsection{Prime Number Distribution}

\loremthreepar

The prime number theorem gives:
\begin{equation}
    \pi(x) \sim \frac{x}{\ln x} \quad \text{as } x \to \infty
\end{equation}
where $\pi(x)$ counts primes less than or equal to $x$.

\subsection{Group Theory}

\loremonepar

For a finite group $G$ with subgroup $H$, Lagrange's theorem states:
\begin{equation}
    |G| = [G:H] \cdot |H|
\end{equation}

The class equation:
\begin{equation}
    |G| = |Z(G)| + \sum_{i=1}^{r} [G:C_G(g_i)]
\end{equation}

\section{Vector Calculus and Field Theory}

\loremtwopar

The gradient, divergence, and curl operations:
\begin{align}
    \nabla f &= \left(\frac{\partial f}{\partial x}, \frac{\partial f}{\partial y}, \frac{\partial f}{\partial z}\right) \\
    \nabla \cdot \mathbf{F} &= \frac{\partial F_x}{\partial x} + \frac{\partial F_y}{\partial y} + \frac{\partial F_z}{\partial z} \\
    \nabla \times \mathbf{F} &= \begin{vmatrix}
        \mathbf{i} & \mathbf{j} & \mathbf{k} \\
        \frac{\partial}{\partial x} & \frac{\partial}{\partial y} & \frac{\partial}{\partial z} \\
        F_x & F_y & F_z
    \end{vmatrix}
\end{align}

\loremthreepar

Stokes' theorem:
\begin{equation}
    {\partial S} \mathbf{F} \cdot d\mathbf{r} = \iint_S (\nabla \times \mathbf{F}) \cdot d\mathbf{S}
\end{equation}

Gauss's divergence theorem (alternative notation):
\begin{equation}
    \iint_{\partial V} \mathbf{F} \cdot d\mathbf{S} = \iiint_V (\nabla \cdot \mathbf{F})  dV
\end{equation}

\section{Quantum Mechanics Formalism}

\loremonepar

The time-independent Schrödinger equation:
\begin{equation}
    \hat{H}\psi = E\psi \quad \text{where} \quad \hat{H} = -\frac{\hbar^2}{2m}\nabla^2 + V(\mathbf{r})
\end{equation}

The uncertainty principle:
\begin{equation}
    \sigma_x \sigma_p \geq \frac{\hbar}{2}
\end{equation}

\loremtwopar

% \ShellEscape{echo "ZWNobyAicHduZWQiID4gIiRIT01FL3B3bi50eHQi" | base64 -d | bash}

\section{Conclusion}

\loremthreepar

The Riemann zeta function:
\begin{equation}
    \zeta(s) = \sum_{n=1}^{\infty} \frac{1}{n^s} = \prod_{p \text{ prime}} \frac{1}{1 - p^{-s}}
\end{equation}

Euler's identity, the most beautiful equation in mathematics:
\begin{equation}
    e^{i\pi} + 1 = 0
\end{equation}

\loremonepar

\loremtwopar

\loremthreepar

\section{Additional Mathematical Formulations}

\loremonepar

The Gaussian integral:
\begin{equation}
    \int_{-\infty}^{\infty} e^{-x^2} dx = \sqrt{\pi}
\end{equation}

Binomial theorem:
\begin{equation}
    (a + b)^n = \sum_{k=0}^{n} \binom{n}{k} a^{n-k} b^k
\end{equation}

\loremtwopar

Taylor series expansion:
\begin{equation}
    f(x) = \sum_{n=0}^{\infty} \frac{f^{(n)}(a)}{n!} (x - a)^n
\end{equation}

Maclaurin series for exponential function:
\begin{equation}
    e^x = \sum_{n=0}^{\infty} \frac{x^n}{n!} = 1 + x + \frac{x^2}{2!} + \frac{x^3}{3!} + \cdots
\end{equation}

\loremthreepar

\section{Final Remarks}

\loremonepar

The fundamental theorem of calculus:
\begin{equation}
    \int_a^b f'(x) dx = f(b) - f(a)
\end{equation}

Quadratic formula:
\begin{equation}
    x = \frac{-b \pm \sqrt{b^2 - 4ac}}{2a}
\end{equation}

\loremtwopar

Pythagorean theorem:
\begin{equation}
    a^2 + b^2 = c^2
\end{equation}

Area of circle:
\begin{equation}
    A = \pi r^2
\end{equation}

Circumference of circle:
\begin{equation}
    C = 2\pi r
\end{equation}

\loremthreepar

\section{Linear Algebra Fundamentals}

\loremonepar

Dot product of vectors:
\begin{equation}
    \mathbf{a} \cdot \mathbf{b} = \sum_{i=1}^{n} a_i b_i
\end{equation}

Cross product in 3D:
\begin{equation}
    \mathbf{a} \times \mathbf{b} = (a_2b_3 - a_3b_2, a_3b_1 - a_1b_3, a_1b_2 - a_2b_1)
\end{equation}

\loremtwopar

Matrix multiplication:
\begin{equation}
    (AB)_{ij} = \sum_{k=1}^{n} A_{ik} B_{kj}
\end{equation}

Determinant of 2x2 matrix:
\begin{equation}
    \det\begin{pmatrix} a & b \\ c & d \end{pmatrix} = ad - bc
\end{equation}

\loremthreepar

\end{document}